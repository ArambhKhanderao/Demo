\section{CE21B021}

\title{\textbf{Assignment 4}} 
\author{\textbf{Arambh Khanderao, CE21B021 }} 
\date{\textbf{July 2022}} 
\maketitle 

\section*{Gibb's Free Energy Change} 

\begin{equation}
\Delta G = \Delta H - T\Delta S 
\end{equation}

Gibbs free energy, also known as the Gibbs function, Gibbs energy, or free enthalpy, is a 
quantity that is used to measure the maximum amount of work done in a thermodynamic system when the temperature and pressure are kept 
constant. Gibbs free energy is denoted by the symbol ‘G’. Its value is usually expressed in Joules or Kilojoules. Gibbs free energy can be 
defined as the maximum amount of work that can be extracted from a closed system. If the reactants and products are all in their 
thermodynamic standard states, then the defining equation is written as 

\begin{equation}
\Delta G^\circ = \Delta H^\circ - T\Delta S^\circ 
\end{equation} 

Gibbs free energy was originally defined graphically. In 1873, American 
scientist \textbf{Willard Gibbs} published his first thermodynamics paper, "Graphical Methods in the Thermodynamics of Fluids", in which 
Gibbs used the two coordinates of the entropy and volume to represent the state of the body. 

\begin{center} 
\begin{tabular}{|c|c|}

\hline \textbf{Symbols} & \textbf{Meaning} \\ 
\hline $\Delta G^\circ$ & Gibb's free energy \\ 
\hline $\Delta H^\circ$ & Change in  enthalpy \\ 
\hline T & Temperature \\ 
\hline $\Delta S^\circ$ & Change in entropy \\ 
\hline
    
\end{tabular} 
\end{center} 
